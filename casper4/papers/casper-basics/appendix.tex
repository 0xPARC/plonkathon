\appendix
\clearpage
\part*{Appendix}

\section{Leak formula derivation}
\label{app:leak}
\todo{Put details of the derivation of the leak formula here.}

\section{Unused text}

\todo{This is where text goes that for which a home hasn't been found yet.  If no home is found, it will be deleted.}



for the same \epoch and \hash as in \eqref{eq:msgPREPARE}.  The $\hash$ is the block hash of the block at the start of the epoch.  A hash $\hash$ being justified entails that all fresh (non-finalized) ancestor blocks are also justified.  A hash $\hash$ being finalized entails that all ancestor blocks are also finalized, regardless of whether they were previously fresh or justified.  An ``ideal execution'' of the protocol is one where, at the start of every epoch, every validator Prepares and Commits the first blockhash of each epoch, specifying the same $\epochsource$ and $\hashsource$.

In the Casper protocol, there exists a set of validators, and in each \textit{epoch} (see below) validators may send two kinds of messages: $$[PREPARE, epoch, hash, epoch_{source}, hash_{source}]$$ and $$[COMMIT, epoch, hash]$$


If, during an epoch $e$, for some specific ancestry hash $h$, for any specific ($epoch_{source}, hash_{source}$ pair), there exist $\frac{2}{3}$ prepares of the form $$[PREPARE, e, h, epoch_{source}, hash_{source}]$$, then $h$ is considered \textit{justified}. If $\frac{2}{3}$ commits are sent of the form $$[COMMIT, e, h]$$ then $h$ is considered \textit{finalized}.


\section{Notes to Authors}
\subsection{Questions}
\begin{itemize}
\item True/False: The Dynasty counter increments iff there's been a finalization?
\end{itemize}


\subsection{Notes on Suggested Terminology}
\begin{itemize}
\item parent $\rightarrow$ predecessor.
\item child $\rightarrow$ successor (unless want to emphasize there can be multiple candidate successors)
\item ancestors $\rightarrow$ lineage
\item to refer to the set of $\{$ predecessor, successor $\}$ $\rightarrow$ adjacent
\end{itemize}


\subsection{Todo}
\begin{itemize}
\item \sout{Reference the various Figures within the text so we more easily know what goes with what.}
\item \todo{fill me in}
\end{itemize}
