% make in NIPS format
\usepackage{nips10submit_e}
\nipsfinalcopy



\usepackage{graphicx}
\DeclareGraphicsExtensions{.pdf,.eps,.png,.jpg}		% search for .pdf, then .eps, then .pngs, then .jpg

% look in these subdirectories for graphics referenced by \includegraphics
% each entry must end with a /
\graphicspath{{figs/}{figures/}{images/}{./}}

\newcommand*{\red}[1]{ \color{red} #1}

\usepackage{tabularx}
\usepackage{url}
\usepackage{cite}
\usepackage{amsmath}

\usepackage{ lmodern }			% I prefer this over times

\usepackage{array}			% replacement for eqnarray.  Must be BEFORE \usepackage{arydshln}
\usepackage{units}			% for \nicefrac{\alpha}{\beta}


\usepackage{amsthm}		% for theorems
\newtheorem{definition}{Definition}

\usepackage{microtype}

%\usepackage{wasysym}

\usepackage{textcomp, marvosym} % pretty symbols
\usepackage{booktabs} 	% for much better looking tables

% for indicator functions
\usepackage{dsfont}



% for pretty Euler script
\usepackage[mathscr]{euscript}



\usepackage{fullpage} % save trees

\usepackage{subfig}
%\usepackage{float} % for \subfloat

%%%%%%%%%%%%%%%%%%%%%%%%%%%%%%%%%%%%%%%%%%
%% More customizable Lists
%%%%%%%%%%%%%%%%%%%%%%%%%%%%%%%%%%%%%%%%%%
% Better symbols custom enumerative lists, define any symbol you'd like
\usepackage{enumitem}


%%%%%%%%%%%%%%%%%%%%%%%%%%%%%%%%%%%%%%%%%%
%% Custom Symbols 
%%%%%%%%%%%%%%%%%%%%%%%%%%%%%%%%%%%%%%%%%%
% \xspace at the end of custom macros never fucks up spacing.
% example of best practice: \newcommand{\apples}{\textsf{AppLeS}\xspace} 
\usepackage{xspace}



%%%%%%%%%%%%%%%%%%%%%%%%%%%%%%%%%%%%%%%%
%% Abbreviations you'll always want
%%%%%%%%%%%%%%%%%%%%%%%%%%%%%%%%%%%%%%%%
\newcommand*{\TODO}[1]{{\centering {\sffamily \color{red} #1} \vskip10pt }}
\newcommand*{\todo}[1]{{\sffamily [{\color{red} #1}]}}
\newcommand*{\fix}[1]{{\sffamily [{\textnormal{\color{red} #1}}]}}



%-----------------------------------------------------------------------------
%  Cross references
%-----------------------------------------------------------------------------
% The following code defines how you make references to figures, tables, etc...
% It is defined in one place only, and can be modified for all references
% in the document at the same time.
% Instead of typing each time: "see Fig. \ref{myfig}" you can create a command
% \figref which will do the job. Then in text you only type \figref{myfig} and LaTeX
% will do the rest.
\newcommand{\tblref}[1]{Table~\ref{#1}}
%\renewcommand*{\figref}[1]{Fig.~\ref{#1}}
\newcommand{\equref}[1]{(\ref{#1})}

\newcommand{\Tblref}[1]{Table~\ref{#1}}
\newcommand{\Figref}[1]{Figure~\ref{#1}}


%%%%%%%%%%%%%%%%%%%%%%%%%%%%%%%%%%%%%%%%


%-----------------------------------------------------------------------------
%  Misc symbols that I like
%-----------------------------------------------------------------------------
\newcommand*{\opname}[1]{\operatorname{#1}}


\renewcommand*{\to}{\rightarrow}


%%%%



